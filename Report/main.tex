\documentclass[a4paper]{article}
\usepackage{a4wide,amssymb,epsfig,latexsym,multicol,array,hhline,fancyhdr}
%\usepackage{vntex}
\usepackage{amsmath}
\usepackage{lastpage}
\usepackage[lined,boxed,commentsnumbered]{algorithm2e}
\usepackage{enumerate}
\usepackage{color}
\usepackage{graphicx}							% Standard graphics package
\usepackage{array}
\usepackage{tabularx, caption}
\usepackage{multirow}
\usepackage{multicol}
\usepackage{rotating}
\usepackage{graphics}
\usepackage{geometry}
\usepackage{setspace}
\usepackage{epsfig}
\usepackage{tikz}
\usetikzlibrary{arrows,snakes,backgrounds}
\usepackage{hyperref}
\hypersetup{urlcolor=blue,linkcolor=black,citecolor=black,colorlinks=true} 
%\usepackage{pstcol} 								% PSTricks with the standard color package
\usepackage{float}
\usepackage[none]{hyphenat}

\newtheorem{theorem}{{\bf Theorem}}
\newtheorem{property}{{\bf Property}}
\newtheorem{proposition}{{\bf Proposition}}
\newtheorem{corollary}[proposition]{{\bf Corollary}}
\newtheorem{lemma}[proposition]{{\bf Lemma}}

\AtBeginDocument{\renewcommand*\contentsname{Contents}}
\AtBeginDocument{\renewcommand*\refname{References}}
%\usepackage{fancyhdr}
\setlength{\headheight}{40pt}
\pagestyle{fancy}
\fancyhead{} % clear all header fields
\fancyhead[L]{
 \begin{tabular}{rl}
    \begin{picture}(25,15)(0,0)
    \put(0,-8){\includegraphics[width=8mm, height=8mm]{hcmut.png}}
    %\put(0,-8){\epsfig{width=10mm,figure=hcmut.eps}}
   \end{picture}&
	%\includegraphics[width=8mm, height=8mm]{hcmut.png} & %
	\begin{tabular}{l}
		\textbf{\bf \ttfamily University of Technology, Ho Chi Minh City}\\
		\textbf{\bf \ttfamily Faculty of Computer Science and Engineering}
	\end{tabular} 	
 \end{tabular}
}
\fancyhead[R]{
	\begin{tabular}{l}
		\tiny \bf \\
		\tiny \bf 
	\end{tabular}  }
\fancyfoot{} % clear all footer fields
\fancyfoot[L]{\scriptsize \ttfamily Software Engineering - Semester 1 (2023 - 2024)}
\fancyfoot[R]{\scriptsize \ttfamily Page {\thepage}/\pageref{LastPage}}
\renewcommand{\headrulewidth}{0.3pt}
\renewcommand{\footrulewidth}{0.3pt}


%%%
\setcounter{secnumdepth}{4}
\setcounter{tocdepth}{3}
\makeatletter
\newcounter {subsubsubsection}[subsubsection]
\renewcommand\thesubsubsubsection{\thesubsubsection .\@alph\c@subsubsubsection}
\newcommand\subsubsubsection{\@startsection{subsubsubsection}{4}{\z@}%
                                     {-3.25ex\@plus -1ex \@minus -.2ex}%
                                     {1.5ex \@plus .2ex}%
                                     {\normalfont\normalsize\bfseries}}
\newcommand*\l@subsubsubsection{\@dottedtocline{3}{10.0em}{4.1em}}
\newcommand*{\subsubsubsectionmark}[1]{}
\makeatother


\begin{document}

\begin{titlepage}
\begin{center}
VIETNAM NATIONAL UNIVERSITY, HO CHI MINH CITY \\
UNIVERSITY OF TECHNOLOGY \\
FACULTY OF COMPUTER SCIENCE AND ENGINEERING
\end{center}

\vspace{1cm}

\begin{figure}[h!]
\begin{center}
\includegraphics[width=3cm]{hcmut.png}
\end{center}
\end{figure}

\vspace{1cm}


\begin{center}
\begin{tabular}{c}
\multicolumn{1}{l}{\textbf{{\Large Machine Learning - CO3001}}}\\
~~\\
\hline
\\
\multicolumn{1}{l}{\textbf{{\Large Assignment}}}\\
\\
\textbf{{\Huge Iris Flower Classification}}\\
\\
\hline
\end{tabular}
\end{center}

\vspace{3cm}

\begin{table}[h]
\begin{tabular}{rrl}
\hspace{3 cm} & \textbf{{\ Instructor}}: & Truong Thi Thai Minh \\
\hspace{3 cm} & \textbf{{\ Class}}: & CC01 - Group 1 \\
\hspace{3 cm} & \textbf{{\ Student(s)}}: 
& Tran Anh Kiet - 2152147 \\ 
& &  Le Duc Hoang Nam - 2111795\\
& &  Vo Thien Nam - 2111817\\

& &   \\
& &   \\
\end{tabular}
\end{table}

\begin{center}
{\footnotesize HO CHI MINH CITY, OCTOBER 2023}
\end{center}
\end{titlepage}


%\thispagestyle{empty}

\newpage
	\tableofcontents
\newpage

\section{Introduction}
Irises influenced the design of the French fleur-de-lis, are commonly used in the Japanese art of flower arrangement known as Ikebana, and underlie the floral scents of the “essence of violet” perfume. They’re also the subject of this well-known machine learning project, in which you must create an ML model capable of sorting irises based on five factors into one of three classes, Iris Setosa, Iris Versicolour, and Iris Virginica.

To get started, the data set below includes 50 instances of each of the three iris classes for a total of 150 instances. While one of the classes is linearly separable, the other two are not. Your task is to create a model capable of classifying each iris instance into the appropriate class based on four attributes: sepal length, sepal width, petal length, and petal width. 

\section{Literature Review}
Discuss previous work and findings related to the Iris classification problem.

\section{Methodology}
\subsection{Data Collection}
The dataset was obtained from the UCI Machine Learning Repository. For more information, visit the \href{https://archive.ics.uci.edu/ml/datasets/iris}{UCI Iris Dataset} page.\\
This is one of the earliest datasets used in the literature on classification methods and widely used in statistics and machine learning.  The data set contains 3 classes of 50 instances each, where each class refers to a type of iris plant.  One class is linearly separable from the other 2; the latter are not linearly separable from each other.

Predicted attribute: class of iris plant.

This is an exceedingly simple domain.

\subsection{Data Preprocessing}
Detail the steps taken to preprocess the data.

\subsection{Model Selection}
Discuss the models you considered for this project.

\subsection{Model Training}
Explain how you trained your model.

\section{Results and Discussion}
\subsection{Model Performance}
Present the results of your model.

\subsection{Analysis}
Analyze and interpret the results.

\section{Conclusion}
Summarize the main findings, the implications of your work, and any future work.

\section{References}
List of the references you used in your report.


\end{document}

